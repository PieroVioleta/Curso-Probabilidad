\documentclass[a4paper,9pt]{report}
\usepackage{anysize} % Soporte para el comando \marginsize
\marginsize{1.2cm}{1.2cm}{1cm}{1cm}
\usepackage{amsfonts}
\usepackage{amssymb}
\usepackage{mathpazo}
\usepackage[latin1]{inputenc}
\usepackage[english,spanish]{babel}
\usepackage{amsmath}
\usepackage{multicol} 
\columnsep=7mm
\usepackage{latexsym}
\usepackage{mathrsfs}
\usepackage{indentfirst}
\usepackage{graphicx}
\usepackage{enumitem}

\usepackage{times}
\usepackage{decimal}
\newcommand{\inner}[2]{\langle{#1},{#2}\rangle}

\newcommand{\cabecera}[1]{\begin{figure}[h]
		\begin{minipage}[c]{0.05\columnwidth}
			\centering\includegraphics[width=2cm]{escudo.pdf}
		\end{minipage}
		\hfill{}
		\begin{minipage}[c]{0.86\columnwidth}
			\centering\flushleft {Universidad Nacional de Ingenier\'ia\\
				Facultad de Ciencias\\
				Escuela Profesional de Matem\'atica \hfill #1}
		\end{minipage}
	\end{figure}\vspace{-0.5cm}
}


\pagestyle{empty}
\begin{document}
	\cabecera{Ciclo 2018 - II}
	
	\begin{center}
		{\Large \textbf{Soluciones del Examen Parcial de C\'alculo de Probabilidades}}\\
		{\Large \textbf{CM-	1H2}}
	\end{center}
	
	\noindent
	\rule{\textwidth}{0.6pt}
	
	
	\vspace{0.3cm}
\begin{enumerate}
\item [] \textbf{Respuesta 1}


\begin{enumerate}
	\item  $\Omega=\{QR,GR,L,GL\}$.
	\item  Se tiene que $\mathbb{P}(QL)=1-\mathbb{P}(QR)-\mathbb{P}(GR)-\mathbb{P}(GL)=0.1$. Luego, \\
	$\mathbb{P}(L)=\mathbb{P}(QL)+\mathbb{P}(GL)=0.1+0.2=0.3$. 
	\item $\mathbb{P}(G)= \mathbb{P}(GR)+\mathbb{P}(GL)=0.2+0.2=0.4$.
	\item $\mathbb{P}(L\cup G)=\mathbb{P}(L)+\mathbb{P}(G)- \mathbb{P}(LG)=0.3+0.4-0.2=0.5$.
\end{enumerate}

\item [] \textbf{Respuesta 2}

 Let X be the largest number selected. Clearly,
 
\[
\mathbb{P}(X =i) = \mathbb{P}(X \leq i) - \mathbb{P}(X \leq i -1) = \biggl(\frac{i}{N}\biggr)^n - \biggl(\frac{i - 1}{N}\biggr)^n, \quad i =1,2, \dots, N. 
\]

As\'i,

\begin{align*}
\mathbb{E}(X) &= \sum_{i =1}^{N}\biggl[\frac{i^{n +1}}{N^n} - \frac{i(i -1)^n}{N^n}\biggr] = \frac{1}{N^n}\sum_{i =1}^{N}[i^{n +1} -i(i -1)^{n}] \\
&= \frac{1}{N^n}\sum_{i =1}^{N}[i^{n +1} -(i -1)^{n + 1} -(i -1)^n] = \frac{N^{n +1} - \sum_{i =1}^{N}(i -1)^{n}}{N^n}. 
\end{align*}


Para un valor grande de $N$,

\[
\sum_{i =1}^{N}( i -1)^n \approx \int_{0}^{N}x^n dx = \frac{N^{n +1}}{n +1}.
\]

Por tanto,

\[
\mathbb{E}(X) \approx \frac{N^{n +1} - \frac{N^{n +1}}{n +1}}{N^n} = \frac{nN}{n + 1}.
\]
\item [] \textbf{Respuesta 3}

Sea A el evento en el que al menos $(n - 1)$ de las monedas de las $n$ monedas muestren todas las caras o todo sellos  y sea B el evento de que todas las $n$ monedas muestren todas caras o todas sellos. Entonces, se pide  encontrar el valor num\'erico de,


\[
\mathbb{P}(B|A) = \frac{\mathbb{P}(A \cap B)}{\mathbb{P}(A)}.
\]

Ahora,

\[
\mathbb{P}(A \cap B) = \mathbb{P}(B) = \biggl (\frac{1}{2}\biggr)^n +  \biggl (\frac{1}{2}\biggr)^{n} =  \biggl (\frac{1}{2}\biggr)^{n -1}.
\]

Adem\'as, el evento A ocurrir\'a si se obtienen exactamente $(n-1)$ caras, o si se obtienen exactamente $(n- 1)$ sellos, o si se obtienen exactamente $n$ caras, o si se obtienen exactamente $n$ sellos, de modo que

\[
\mathbb{P}(A) =  n\biggl (\frac{1}{2}\biggr)^n +  n\biggl (\frac{1}{2}\biggr)^{n} +  \biggl (\frac{1}{2}\biggr)^{n} +  \biggl (\frac{1}{2}\biggr)^{n -1}=  (1 + n)\biggl (\frac{1}{2}\biggr)^{n -1}.
\]

As\'i,

\[
\mathbb{P}(B|A) = \frac{(1/2)^{n -1}}{(1 +n)(1/2)^{n -1}} = \frac{1}{n + 1}, \quad n =3, 4,\dots, \infty.
\]

\item [] \textbf{Respuesta 4}

\begin{enumerate}
	\item  El valor esperado de $X$ es
	\begin{eqnarray*}
		\mathbb{E}(X)&=&\displaystyle \sum_{x=0}^{4}x\mathbb{P}_X(x)\\
		&=&0C_0^4\frac{1}{2^4}+1C_1^4\frac{1}{2^4}+2C_2^4\frac{1}{2^4}+3C_3^4\frac{1}{2^4}+4C_4^4\frac{1}{2^4}\\
		&=&[4+12+12+4]\frac{1}{2^4}\\
		&=& 2,
	\end{eqnarray*}
	\\
	El valor esperado de $X^2$ es,
	
	\begin{eqnarray*}
		\mathbb{E}(X^2)&=&\displaystyle \sum_{x=0}^{4}x^2\mathbb{P}_X(x)\\
		&=&0^2C_0^4\frac{1}{2^4}+1^2C_1^4\frac{1}{2^4}+2^2C_2^4\frac{1}{2^4}+3^2C_3^4\frac{1}{2^4}+4^2C_4^4\frac{1}{2^4}\\
		&=&[4+24+36+16]\frac{1}{2^4}\\
		&=& 5,
	\end{eqnarray*}
	\\
	La varianza de $X$ es 
	$$\text{Var}(X)=\mathbb{E}(X)^2-\mathbb{E}(X^2)=5-2^2=1.$$
	Luego, la desviaci\'on est\'andar de $X$ es $\sigma_X=\sqrt{\text{Var}(X)}=1$.
	\item La probabilidad de que $X$ est\'e dentro de una desviaci\'on est\'andar de su valor esperado es
	\begin{eqnarray*}
		\mathbb{P}(\mu_X-\sigma_X\leq X \leq \mu_X+\sigma_X)	&=& \mathbb{P}(2-1\leq X \leq 2+1)\\
		&=& \mathbb{P}(1\leq X \leq 3).
	\end{eqnarray*}
	Luego, 
	$$\mathbb{P}(1\leq X \leq 3) = \mathbb{P}_X(1)+\mathbb{P}_X(2)+\mathbb{P}_X(3)=\frac{7}{8}.$$
\end{enumerate}


\end{enumerate}
\end{document}

		