\documentclass[a4paper,11pt]{report}
\usepackage{anysize} % Soporte para el comando \marginsize
\marginsize{1.2cm}{1.2cm}{1cm}{1cm}
\usepackage{amsfonts}
\usepackage{amssymb}
\usepackage{mathpazo}
\usepackage[latin1]{inputenc}
\usepackage[english,spanish]{babel}
\usepackage{amsmath}
\usepackage{multicol} 
\columnsep=7mm
\usepackage{latexsym}
\usepackage{mathrsfs}
\usepackage{indentfirst}
\usepackage{graphicx}
\usepackage{enumitem}

\usepackage{times}

\newcommand{\inner}[2]{\langle{#1},{#2}\rangle}

\newcommand{\cabecera}[1]{\begin{figure}[h]
		\begin{minipage}[c]{0.05\columnwidth}
			\centering\includegraphics[width=2cm]{escudo.pdf}
		\end{minipage}
		\hfill{}
		\begin{minipage}[c]{0.86\columnwidth}
			\centering\flushleft {Universidad Nacional de Ingenier\'ia\\
				Facultad de Ciencias\\
				Escuela Profesional de Matem\'atica \hfill #1}
		\end{minipage}
	\end{figure}\vspace{-0.5cm}
}


\pagestyle{empty}
\begin{document}
	\cabecera{Ciclo 2018 - II}
	
	\begin{center}
		{\Large \textbf{Examen Parcial C\'alculo de Probabilidades}}\\
		{\Large \textbf{CM-	1H2}}
	\end{center}
	
	\noindent
	\rule{\textwidth}{0.6pt}
	
	
	\vspace{0.3cm}
	
\begin{enumerate}
\item Los programas de computadoras son clasificados por la longitud del c\'odigo fuente y el tiempo de ejecuci\'on; los programas con m\'as de $150$ l\'ineas de c\'odigo fuente son llamados GRANDES (G) y los programas con $\leq$150 l\'ineas de  c\'odigo son llamados PEQUE\~NOS (Q). Asimismo, los programas  R\'APIDOS (R) son aquellos que se ejecutan en menos de $0.1$ segundos y los programas LENTOS (L) son aquellos que requieren al menos 0.1 segundos para ejecutarse. 
	
Al monitorear un programa ejecutado por una computadora, se observe la longitud del c\'odigo fuente y el tiempo de ejecuci\'on de este. Luego, se plantea un modelo de probabilidad para este experimento, el cual contiene la siguiente informaci\'on: $\mathbb{P}(QR)=0.5$, $\mathbb{P}(GR)=0.2$ y $\mathbb{P}(GL)=0.2$.
	\begin{enumerate}
		\item ?`Cu\'al es el espacio muestral para este experimento? \hfill{(1 pto.)}
		\item ?`Cu\'al es la probabilidad de que un programa sea LENTO?\hfill{(2 pto.)} 
		\item ?`Cu\'al es la probabilidad de que un programa sea GRANDE?\hfill{(1 pto.)}
		\item ?`Cu\'al es la probabilidad de que un programa sea LENTO o GRANDE?\hfill{(1 pto.)}
	\end{enumerate}
\item Suppose that n random integers are selected from $\{1, 2, \dots, N\}$ with replacement. What is the expected value of the largest number selected? Show that for large $N$  the answer is approximately $nN/(n + 1)$. \hfill{(5 ptos.)}

\item Supongamos que $n (\geq 3)$ monedas justas (o insesgadas) se lanzan simult\'aneamente. Dado que al menos $(n-1)$ monedas de las $n$ monedas muestran todas  caras o todas sellos, encuentra una expresi\'on expl\'icita (en funci\'on de $n$) para la probabilidad de que todas las $n$ monedas muestren todas  caras o todos  sellos.\hfill{(5 ptos.)}

\item Dada la variable aleatoria discreta $X$ tal que $\displaystyle P_X(x)=C_x^4\big(\frac{1}{2}\big)^4 $.
\begin{enumerate}
	\item Encuentra la desviaci\'on est\'andar de $X$.\hfill{(3 ptos.)}
	\item Calcula $\mathbb{P}(\mu_X-\sigma_X\leq X \leq \mu_X+\sigma_X)$\hfill{(2 ptos.)}
\end{enumerate}


\end{enumerate}
\end{document}

		