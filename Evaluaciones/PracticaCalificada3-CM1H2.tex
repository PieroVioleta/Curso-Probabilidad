\documentclass[a4paper,10pt]{report}
\usepackage{anysize} % Soporte para el comando \marginsize
\marginsize{1.2cm}{1.2cm}{1cm}{1cm}
\usepackage{amsfonts}
\usepackage{amssymb}
\usepackage{mathpazo}
\usepackage[latin1]{inputenc}
\usepackage[english,spanish]{babel}
\usepackage{amsmath}
\usepackage{multicol} 
\columnsep=7mm
\usepackage{latexsym}
\usepackage{mathrsfs}
\usepackage{indentfirst}
\usepackage{graphicx}
\usepackage{enumitem}

\usepackage{times}

\newcommand{\inner}[2]{\langle{#1},{#2}\rangle}

\newcommand{\cabecera}[1]{\begin{figure}[h]
		\begin{minipage}[c]{0.05\columnwidth}
			\centering\includegraphics[width=2cm]{escudo.pdf}
		\end{minipage}
		\hfill{}
		\begin{minipage}[c]{0.86\columnwidth}
			\centering\flushleft {Universidad Nacional de Ingenier\'ia\\
				Facultad de Ciencias\\
				Escuela Profesional de Matem\'atica \hfill #1}
		\end{minipage}
	\end{figure}\vspace{-0.5cm}
}


\pagestyle{empty}
\begin{document}
	\cabecera{Ciclo 2018 - II}
	
	\begin{center}
		{\Large \textbf{Pr\'actica calificada de C\'alculo de Probabilidades}}\\
		{\Large \textbf{CM-	1H2}}
	\end{center}
	
	\noindent
	\rule{\textwidth}{0.6pt}
	
	
	\vspace{0.3cm}
	
\begin{enumerate}
\item Sea $X$ una variable aleatoria con distribuci\'on binomial $\text{Binomial}(n,p)$. Si fijamos $\lambda=np$ (constante) para luego hacer $n\to\infty$ y $p\to0$, comprueba entonces que la distribuci\'on binomial se aproxima a la distribuci\'on de Poisson con par\'ametro $\lambda$. \hfill{(4 ptos.)}
\item Decir si son verdaderas o falsas las siguientes afirmaciones. En caso de que sean verdaderas demostrarlo y en caso de que sean falsas dar un contraejemplo. 
\begin{enumerate}
\item Para cualquier variable aleatoria $X$  se verifica que $\forall a,b\in\mathbb{R}$ se tiene que
	$$\mathbb{P}(a<X\leq b)=\mathbb{P}(a<X< b)=\mathbb{P}(a\leq X\leq b)=\mathbb{P}(a\leq X< b).\eqno{\text{(1 pto.)}}$$
\item Si $X$ es una variable aleatoria discreta o continua entonces se verifica, si $a < b < c$, que $\mathbb{P}(a \leq  X \leq c) = \mathbb{P}(a \leq  X \leq b)+ \mathbb{P}(b \leq  X \leq c)$.\hfill{(1 pto.)}
\item Las condiciones sobre $f(x)$ en el caso de las variables aleatorias discretas de que se cumple  $0 \leq f (x) \leq 1$ y que $\displaystyle\sum_x f(x) = 1$ se traduce en el caso de las variables aleatorias continuas de que  $f(x)$  cumple  que $0 \leq f (x) \leq 1$ y que $\int f(x)dx = 1$.\hfill{(1 pto.)}
\item Sea $X$ una variable aleatoria  con funci\'on de densidad dada por
	$$f(x)=\left\{ \begin{matrix}\frac{x}{2}&\text{si }0<x<2\\
	0&\text{en otros casos}\end{matrix}\right.$$
	entonces su funci\'on de distribuci\'on viene dada por
	$$F(x)=\left\{ \begin{matrix}\frac{x^2}{4}&\text{si }0<x<2\\
	0&\text{en otros casos}\end{matrix}\right.\eqno{\text{(1 pto.)}}$$
\end{enumerate}
\item Suppose that an aircraft engine will fail in flight with probability $1 -p$ independently of the plane's other engines. Also suppose that a plane can complete the
journey successfully if at least half of its engines do not fail.

\begin{enumerate}
\item Is it true that a four-engine plane is always preferable to a two-engine plane?. Explain. \hfill{(2 ptos.)}
\item Is it true that a five-engine plane is always preferable to a three-engine plane?
Explain. \hfill{(2 ptos.)}
\end{enumerate}
\item Supongamos que $n$ pasas se mezclan completamente en una masa para galletas. Si elaboramos $k$ galletas de pasas de uva de igual tama\~no a partir de esta mezcla, ?`cu\'al es la probabilidad de que una galleta dada contenga al menos una pasa?. \hfill{(4 ptos.)}
\item Let $f$ be the probability density function of a random variable $X$. In terms of $f$ ,
 calculate the probability density function of $X^2$. \hfill{(4 ptos.)}
\end{enumerate}
\end{document}

		